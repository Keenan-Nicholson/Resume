\documentclass[]{Keenan-Nicholson-Resume}
\fullname{Keenan Nicholson}
\jobtitle{Software Developer \& GIS Analyst}

\begin{document}
\resumeheader
{\linkedin{kbnicholson/}{linkedin.com/in/kbnicholson/}}
{\email{keenanbnicholson@gmail.com}{keenanbnicholson@gmail.com}}
{\github{Keenan-Nicholson}{github.com/Keenan-Nicholson}}
{\phone{+1 709-649-8326}{+1 709-649-8326}}
{\website{keenannicholson.me/}{keenannicholson.me/}}

\begin{section}{Education}
    \begin{subsection}{Bachelor of Arts}{Major in Computer Science, minor in Geography}{Memorial University of Newfoundland}{Jan 2021 -- Dec 2023}
    \end{subsection}
    \begin{subsection}{Bachelor of Science}{Major in Computational Mathematics}{Memorial University of Newfoundland}{Sep 2016 -- Apr 2020}
    \end{subsection}
\end{section}

\begin{section}{Work Experience (selected)}
    \begin{subsection}{C-CORE}{Image Analyst}{Apr 2023 -- Sep 2023}{St. John's, NL}
        \item Data driven full-stack development of web applications
        \vspace{-4pt}
            \begin{itemize}[itemsep=-6.5pt]
                \item Frontend: React.js, HTML, CSS, Material UI (MUI), and Chakra UI.
                \item Backend: Python, Pandas, SpatioTemporal Asset Catalogs (STAC)
            \end{itemize}
        \vspace{1pt}
        \item Analyzing satellite imagery using technologies such as ArcGIS Pro, Python, Pandas, and Matplotlib to \newline monitor iceberg and river ice activity.
    \end{subsection}

    \begin{subsection}{Memorial University of Newfoundland}{Teaching Assistant}{Sep 2017 -- May 2020}{Corner Brook, NL}
        \item Assisting with the instruction of first year math labs
        \item invigilating quizzes
        \item grading assignments and quizzes for first and second year math courses.
    \end{subsection}
    
    \begin{subsection}{Memorial University of Newfoundland}{Research Assistant}{Jan 2019 -- May 2019}{Corner Brook, NL}
        \item I received a full-time work grant from NSERC. During that time, I researched the properties of algebraic groups as codes and developed an algorithm to decode twisted permutation codes. These codes demonstrate improved reliability and reduced minimum distance compared to other codes in specific scenarios.
    \end{subsection}
\end{section}

\begin{section}{Projects}
\begin{subsectionnobullet}{Canadian Emissions Analysis and Prediction}{Coursework Project}{Nov 2022}{}
    \item{I analyze vehicle fuel efficiency and CO2 emissions using machine learning, studying the impact of factors like transmission type, fuel type, and engine size. Through supervised and unsupervised learning techniques, along with scaling methods, I build a predictive model that estimates a vehicle's efficiency and emissions based on its attributes.}
\end{subsectionnobullet}

\begin{subsectionnobullet}{ISS Tracker Twitter Bot}{Personal Project}{May 2022}{}
    \item{This application implements Cheerio and node-fetch libraries to scrape real-time International Space Station location data from NASA's website. It then formats and utilizes the Twit library to tweet the ISS visibility details for specific cities in Newfoundland and Labrador.}
\end{subsectionnobullet}

\end{section}

\sectiontable{Technical skills}{
    \entry{Programming Languages}{JavaScript / TypeScript, Python, HTML, CSS}
    \entry{Frameworks}{React.js, Chakra UI, Material UI}
    \entry{Operating Systems}{Windows, MAC, Linux}
}

\end{document}
